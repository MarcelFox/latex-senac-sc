\begin{justify}
    Segundo Rachid (2019), a computação em nuvem não se enquadra em ser apenas um
    pedaço de tecnologia como um celular ou um computador, ela nada mais é do que um
    sistema online que disponibiliza serviços ao usuário usando como meio de distribuição a
    internet e se divide em três braços: infraestrutura como serviço, software como serviço e
    plataforma como serviço.

    A computação em nuvem utiliza hardware (parte física dos recursos de computação) e
    software (parte de digital, de controle das partes físicas) para entregar serviços que podem
    ser acessados de inúmeros dispositivos conectados à rede. Essa computação sob demanda
    visa compartilhar recursos de computação como processamento, 
    armazenamento e
    serviços. Alguns desses serviços podem ser de compartilhamento de vídeos como
    “NETFLIX”, com seu catálogo de filmes e séries, o google com o youtube e o gmail, e
    muitos outros serviços acessados hoje pela internet, recursos do cloud computing.

    O que uma empresa necessita para funcionar? Armazenamento de dados, integração com
    filiais, gerenciamento de recursos, gerenciamento de pessoas, essa integração pode ocorrer
    hoje por dois meios, a empresa pode ter sua própria estrutura de recursos como softwares e
    hardwares para implementação desses requisitos de funcionamento, ou pode ser pela
    nuvem. A ideia inicial de serviços pela nuvem é a segurança dos dados, rapidez de colocar
    tudo para funcionar e menor custo de implementação.

    \vspace{0.2cm}
    \begin{flushright}
        \begin{minipage}{.7\textwidth}
            \noindent
            \footnotesize
            \setstretch{1.0}
            
            \begin{justify}
            Bem, podemos dizer que a Computação em Nuvem é um termo para descrever um
            ambiente de computação baseado em uma imensa rede de servidores, sejam estes
            virtuais ou físicos. Uma definição simples pode então ser “um conjunto de recursos
            como capacidade de processamento, armazenamento, conectividade, plataforma,
            aplicações e serviços disponibilizados na internet”. O resultado é que a nuvem pode
            ser vista como o estágio mais evoluído do conceito de virtualização, virtualização do
            próprio data center. \cite{taurion2009cloud}
            \end{justify}
        \end{minipage}
    \end{flushright}
    \vspace{0.2cm}

\end{justify}


\begin{justify}
    Os recursos de hardware físico geralmente não são baratos de se implementar, onde então
    entra a nuvem, à qual oferece inúmeros serviços online como processamento de dados,
    armazenamento, conectividades e até sistemas completos para empresa funcionar, que
    ficam disponíveis 24h por dia, 7 dias por semana em servidores físicos ou servidores
    digitais. \cite{rashid2019cloud}
    
    \subsection{INFRAESTRUTURA}
    No quesito de infraestrutura a computação em nuvem disponibiliza diversos modelos de
    hardware para uso online, ela utiliza a virtualização de recursos físicos e os disponibiliza
    para o consumidor, fazendo com que recursos físicos sejam dinamicamente convertidos em
    recursos lógicos de acordo com o uso. Tais recursos podem ser de processamento (CPU),
    memória (RAM ou de armazenamento), completos sistemas operacionais (windows, linux) e
    softwares de aplicação. \cite{pedrosa2011computaccao}

    Os benefícios de utilizar esse tipo de serviço são inúmeros, por exemplo o custo desses
    serviços online são bem mais baratos do que adquirir esses recursos físicos, os usuários
    pagam apenas aquilo que quiserem usar e o usuários pode optar por upgrades e
    downgrades nos serviços a qualquer momento.
    
    \subsection{PLATAFORMA}
    Com o serviço de plataforma o cloud pode hospedar tanto o software usado quanto às
    especificações de hardware necessárias para rodar tal aplicação, usando assim pouco
    requerimento de hardware e software do usuário, gerando assim uma facilidade de acesso
    de diversos dispositivos a aplicação. Conforme o usuário “compra” o direito de usar uma
    aplicação ele já paga por todo o requerimento necessário para rodar a aplicação.

    Os benefícios desse tipo de serviço são várias, por exemplo uma empresa pode comprar
    um software gerenciado por cloud e com isso ter segurança e acesso de qualquer unidade
    ao mesmo banco de dados, sem a necessidade de investir no seu próprio banco de dados e
    com um investimento mínimo de equipamentos para rodar essas aplicações direto na
    nuvem.\cite{rashid2019cloud}

    \subsection{SOFTWARE}
    Os serviços disponibilizados pela cloud podem ser diversos, desde o gmail da google até
    um sistema inteiro de gerenciamento de uma multinacional. Cada provedor de serviços via
    nuvem se responsabiliza pela segurança dos dados do seu cliente e por manter o serviço
    online, atualizações e problemas encontrados pelos usuários são de responsabilidade do
    próprio provedor do serviço. \cite{taurion2009cloud}

    Essa área da cloud é a mais utilizada pelo usuário final. Serviços de streaming de vídeo,
    jogos e qualquer outra aplicação que o usuário acesse de seu smartphone, computador ou
    qualquer dispositivo ligado a rede de internet, são mantidos online por algum responsável
    por distribuir tal tipo de serviço.

    \subsection{RECUPERAÇÃO}
    Cada serviço disponibilizado pela computação em nuvem, tem um sistema de backup, que
    armazena e faz backups dos dados dos usuários, fazendo com que o usuário não perca
    nenhum de seus dados e arquivos. \cite{rashid2019cloud}

    \subsection{JOGOS DIGITAIS}
    Segundo Huizinga (2003) os jogos são uma atividade lúdica que se utiliza de fenômenos
    físicos, psicológicos e é o ato voluntário de retirar o participante da vida real, adicionando
    uma tensão ao participante por não saber como será o desfecho do jogo.

    Um jogo apresenta quatro pilares fundamentais para sua criação, a representação do jogo
    que oferece ao jogar um mundo completo sem a necessidade de depender de algo do
    exterior e um completo conjunto de regras e definições para ser jogado, a interação onde o
    jogador pode realizar ações que alteram o desfecho do jogo e pode analisar o que sua ação
    resultou, o conflito que são ações que o criador do jogo criou para dificultar com que o
    jogador conclua de forma fácil o jogo tais como barreiras a serem quebradas ou um
    cronômetro e por fim a segurança que o jogador está seguro de riscos físicos de suas ações
    no jogo nao quer dizer que o jogador sairá ileso porém pode perder benefícios dentro do
    jogo. \cite{crawford1984art}



\end{justify}