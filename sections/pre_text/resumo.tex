\thispagestyle{empty}
\begin{Center}
    {\center \bf \MakeUppercase{resumo}}
\end{Center}
    
\begin{justify}
    O estudo apresentado retrata, através de pesquisas em artigos publicados referentes ao tema principal,
     como a utilização da Computação em Nuvem abre espaço à diversos modelos Service as a Service (SaaS), 
     que possibilita diversos modelos de serviço à serem definidos e incorporados junto à nuvem, 
     onde possibilita grandes empresas do mundo dos games como a Sony, Microsoft, Nintendo dentre outras, 
     a explorarem um novo nicho de inovação, que batalha para diminuir custos ao cliente final, de forma à 
     expandir a lucratividade de forma mais horizontal, uma vez que para acessar tal tipo de serviço, o 
     cliente final não precisa comprar uma mídia física para conseguir acessar diversos títulos de jogos. 
     Conclui-se que o formato “X” as a Service será uma grande tendência em diversas áreas e é possível 
     que esta revolução chegue a tal ponto, que o cliente final sequer necessite de um console para ter 
     acesso à jogos com necessidades gráficas avançadas, em um futuro onde o entretenimento dos videogames 
     poderá ser acessado como um outro serviço qualquer, sem a complexidade de se compreender a infraestrutura 
     necessária para que tal serviço seja possível e principalmente, acessado por diversos tipos de classes sociais.
\end{justify}
{Palavras-Chave: \KEYWORDS}
\vspace*{\fill}