\documentclass[a4paper,12pt]{article}
    \usepackage[utf8]{inputenc}
    \usepackage[brazil]{babel}
    \usepackage[
        lmargin=3cm,
        tmargin=3cm,
        rmargin=2cm,
        bmargin=2cm,
        ]{geometry}
    \usepackage{indentfirst}
    \usepackage{graphicx}
        \graphicspath{{./images}}
    
    \usepackage{helvet}
        \renewcommand{\familydefault}{\sfdefault}
    
    \usepackage[document]{ragged2e}
    \usepackage{setspace}
    \usepackage{lipsum}
    \usepackage[none]{hyphenat}

    
    % ---------
    % VARIABLES:
    \newcommand{\ACADEMY}{
        Faculdade Senac Blumenau
    }
    \newcommand{\COURSE}{
        Graduação em SEU CURSO
    }
    \newcommand{\AUTHORS}{
        Nome Sobrenome \\
        Nome Sobrenome 
    }
    \newcommand{\SURNAMES}{
        SOBRENOME, Nome / SOBRENOME, Nome
    }
    \newcommand{\TEACHER}{
        NOME PROFESSOR
    }
    \newcommand{\DOCTITLE}{
        como a computação em nuvem irá \\
        revolucionar o mundo dos games
    }
    \newcommand{\CITY}{
        Blumenau
    }
    \newcommand{\YEAR}{
        2022
    }
    \newcommand{\FULLDATE}{
        \CITY, 10 de Abril de \YEAR
    }
    \newcommand{\PREAMBLE}{
        \singlespacing
        \normalsize \justify Trabalho apresentado à \ACADEMY como requisito
        parcial para obtenção do título de Graduação em \COURSE.
    }
    \newcommand{\KEYWORDS}{
        1. Computação. 2. Nuvem. 3. Games. 4. Internet.
    }
    \newcommand{\DOCTYPE}{
        Trabalho Acadêmico
    }
    \newenvironment{fichacatalografica}
    {
        \thispagestyle{empty}
        \begin{singlespacing}
            \footnotesize
        \end{singlespacing}
    }
    % ---
    
    \title \DOCTITLE
    \author \SURNAMES
    \date \FULLDATE
    \begin{document}
        \pagenumbering{arabic}
        \thispagestyle{empty}
\linespread{1.5}
\begin{Center}
    \MakeUppercase{
        \bf \ACADEMY
        }
    \\
    
    \bf \COURSE
    \\

    \vspace*{1.5cm}
    \bf \AUTHORS
    \\
    
    \vspace*{5cm}
    \bf \MakeUppercase \DOCTITLE
    \\
    
    \vspace*{\fill}
    \bf \CITY \\
    \bf \YEAR
\end{Center}
        
\thispagestyle{empty}
\linespread{1.5}
\begin{Center}
    {\bf \AUTHORS}
    \\
    
    \vspace*{1.5cm}
    {\bf \MakeUppercase \DOCTITLE}
    \\
    
    % PREAMBLE
    % --------
    \vspace*{5cm}
    \hspace{.45\textwidth}
    \begin{minipage}{.5\textwidth}
        \PREAMBLE
        \singlespacing
        {\justify Orientador: \TEACHER}
     \end{minipage}
     % --------
    
    \vspace*{\fill}
    \bf \CITY \\
    \bf \YEAR
\end{Center}
        \begin{fichacatalografica}
    \linespread{1.0}
	\sffamily
	\vspace*{\fill}
	\begin{FlushLeft}
	\fbox{\begin{minipage}[c][8cm]{14.5cm}
	\small
	\AUTHORS
	
	\hspace{0.5cm} \DOCTITLE  / \SURNAMES. --
	\CITY, \FULLDATE-
	
	\hspace{0.5cm} 20 p. : il. (algumas color.) ; 30 cm.\\
	
	\hspace{0.5cm} Orientador:~\TEACHER\\
	
	\hspace{0.5cm}
	\parbox[t]{\textwidth}{\DOCTYPE~--~\ACADEMY,
	\FULLDATE.}\\
	\singlespacing
	\hspace{0.5cm}
        \KEYWORDS
		I. \TEACHER.
		II. \ACADEMY.
		III. \COURSE.
		IV. \DOCTITLE 			
	\end{minipage}}
	\end{FlushLeft}
\end{fichacatalografica}
        \thispagestyle{empty}
\linespread{1.5}
\begin{Center}
    {\bf \AUTHORS}
    \\
    
    \vspace*{1.5cm}
    {\bf \MakeUppercase \DOCTITLE}
    \\

    % PREAMBLE
    % --------
    \vspace*{4cm}
    \hspace{.45\textwidth}
    \begin{minipage}{.5\textwidth}
        \PREAMBLE
     \end{minipage}
     % --------
    
    \vspace*{4.5cm}
    \noindent\rule{6cm}{0.4pt}\\
    \TEACHER
    
    \vspace*{\fill}
    {\FULLDATE}
\end{Center}
        \thispagestyle{empty}
\begin{Center}
    {\center \bf \MakeUppercase{resumo}}
\end{Center}
    
\begin{justify}
    O estudo apresentado retrata, através de pesquisas em artigos publicados referentes ao tema principal,
     como a utilização da Computação em Nuvem abre espaço à diversos modelos Service as a Service (SaaS), 
     que possibilita diversos modelos de serviço à serem definidos e incorporados junto à nuvem, 
     onde possibilita grandes empresas do mundo dos games como a Sony, Microsoft, Nintendo dentre outras, 
     a explorarem um novo nicho de inovação, que batalha para diminuir custos ao cliente final, de forma à 
     expandir a lucratividade de forma mais horizontal, uma vez que para acessar tal tipo de serviço, o 
     cliente final não precisa comprar uma mídia física para conseguir acessar diversos títulos de jogos. 
     Conclui-se que o formato “X” as a Service será uma grande tendência em diversas áreas e é possível 
     que esta revolução chegue a tal ponto, que o cliente final sequer necessite de um console para ter 
     acesso à jogos com necessidades gráficas avançadas, em um futuro onde o entretenimento dos videogames 
     poderá ser acessado como um outro serviço qualquer, sem a complexidade de se compreender a infraestrutura 
     necessária para que tal serviço seja possível e principalmente, acessado por diversos tipos de classes sociais.
\end{justify}
{Palavras-Chave: \KEYWORDS}
\vspace*{\fill}

        \tableofcontents
        \thispagestyle{empty}
        \newpage
        \section{Introdução}
            \newpage
            \subsection{Sub-capítulo}
                \newpage
        \section{Objetivos}
        \newpage
        \section{Fundamentação Teórica}
        \newpage
        \section{Metodologia}
        \newpage
        \section{Referências}
    \end{document}
